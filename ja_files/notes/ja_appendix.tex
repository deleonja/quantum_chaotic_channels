% !TEX root = main.tex
\section{Choi} \label{app:choi}
\begin{align}
%{
\label{eq:appendix:A:1}
\Tr[\mcD^2(t)] &= 
\frac{1}{4}
\qty(
\sum_{i,j,p}
\abs{\matrixel{i, \psi_{0E}}
{U^\dagger(t) \big(\dyad{p} \otimes \one_E\big) U(t)}
{j, \psi_{0E}}
}^2 +
\sum_{i,j,p\ne q}
\abs{\matrixel{i, \psi_{0E}}
{U^\dagger(t) \big(\dyad{p}{q} \otimes \one_E\big) U(t)}
{j, \psi_{0E}}
}^2
) \\
%}
%{
\label{eq:appendix:A:2}
&= 
\frac{1}{4}
\Bigg(
\sum_{i,j}
\abs{\matrixel{i, \psi_{0E}}
{U^\dagger(t) \big(\one_S \otimes \one_E\big) U(t)}
{j, \psi_{0E}}
}^2 
\nonumber \\
&\hspace*{10mm}
- 2 \sum_{i,j}
\Re(
\matrixel{i, \psi_{0E}}
{U^\dagger(t) \big(\dyad{0}{0} \otimes \one_E\big) U(t)}
{j, \psi_{0E}}
\matrixel{j, \psi_{0E}}
{U^\dagger(t) \big(\dyad{1}{1} \otimes \one_E\big) U(t)}
{i, \psi_{0E}}
)
\nonumber \\
& \hspace*{10mm}
+ \sum_{i,j,p\ne q}
\abs{\matrixel{i, \psi_{0E}}
{U^\dagger(t) \big(\dyad{p}{q} \otimes \one_E\big) U(t)}
{j, \psi_{0E}}
}^2
\Bigg) \\
%}
%{
\label{eq:appendix:A:3}
&= 
\frac{1}{2} 
\Bigg(
1 - 
\sum_{i,j}
\Big( \Re(
\matrixel{i, \psi_{0E}}
{U^\dagger(t) \big(\dyad{0}{0} \otimes \one_E\big) U(t)}
{j, \psi_{0E}}
\matrixel{j, \psi_{0E}}
{U^\dagger(t) \big(\dyad{1}{1} \otimes \one_E\big) U(t)}
{i, \psi_{0E}}
)
\nonumber \\
& \hspace*{10mm}
+
\abs{\matrixel{i, \psi_{0E}}
{U^\dagger(t) \big(\dyad{0}{1} \otimes \one_E\big) U(t)}
{j, \psi_{0E}}
}^2
\Big)
\Bigg) \\
%}
%{
\label{eq:appendix:A:4}
&= 
\frac{1}{2} 
\Bigg(
1 - 
\sum_{i,j}
\Big(
\matrixel{i, \psi_{0E}}
{U^\dagger(t) \big(\dyad{0}{0} \otimes \one_E\big) U(t)}
{j, \psi_{0E}}
\matrixel{j, \psi_{0E}}
{U^\dagger(t) \big(\dyad{1}{1} \otimes \one_E\big) U(t)}
{i, \psi_{0E}}
\nonumber \\
& \hspace*{10mm}
+
\abs{\matrixel{i, \psi_{0E}}
{U^\dagger(t) \big(\dyad{0}{1} \otimes \one_E\big) U(t)}
{j, \psi_{0E}}
}^2
\Big)
\Bigg)
%}
\end{align}
\janote{trabajando en analytic\_choi\_check.nb}
\eref{eq:choi:chaometer:purity:2}
\begin{itemize}
\item regular: Segundo y tercer término aproximadamente cero para $t\gg 1$
\item caótico: tercer término igual a cero para $t\gg 1$.
\end{itemize}

\janote{numéricamente parece que podemos hacer esto:}
\begin{align}
%{
\Tr[\mcD^2(t)] &= 
\frac{1}{2} 
\Bigg(
1 - 
\sum_{i}
\underbrace{
\matrixel{i, \psi_{0E}}
{U^\dagger(t) \big(\dyad{0}{0} \otimes \one_E\big) U(t)}
{i, \psi_{0E}}
\matrixel{i, \psi_{0E}}
{U^\dagger(t) \big(\dyad{1}{1} \otimes \one_E\big) U(t)}
{i, \psi_{0E}}
}_{(1 - r_z^2)/4}
\nonumber \\
& \hspace*{12mm}
- 2 \matrixel{0, \psi_{0E}}
{U^\dagger(t) \big(\dyad{0}{0} \otimes \one_E\big) U(t)}
{1, \psi_{0E}}
\matrixel{1, \psi_{0E}}
{U^\dagger(t) \big(\dyad{1}{1} \otimes \one_E\big) U(t)}
{0, \psi_{0E}}
\nonumber \\
& \hspace*{12mm}
+
\sum_{i,j}
\abs{\matrixel{i, \psi_{0E}}
{U^\dagger(t) \big(\dyad{0}{1} \otimes \one_E\big) U(t)}
{j, \psi_{0E}}
}^2
\Bigg) \\
%}
%{	
&= \frac{1}{2} 
\Bigg(
1 - 
\sum_{i}
\underbrace{
\matrixel{i, \psi_{0E}}
{U^\dagger(t) \big(\dyad{0}{0} \otimes \one_E\big) U(t)}
{i, \psi_{0E}}
\matrixel{i, \psi_{0E}}
{U^\dagger(t) \big(\dyad{1}{1} \otimes \one_E\big) U(t)}
{i, \psi_{0E}}
}_{(1 - r_z^2)/4}
\nonumber \\
& \hspace*{12mm}
+ 2 \abs{\matrixel{0, \psi_{0E}}
{U^\dagger(t) \big(\dyad{0}{0} \otimes \one_E\big) U(t)}
{1, \psi_{0E}}
}^2
%\nonumber \\
%& \hspace*{12mm}
+
\sum_{i,j}
\abs{\matrixel{i, \psi_{0E}}
{U^\dagger(t) \big(\dyad{0}{1} \otimes \one_E\big) U(t)}
{j, \psi_{0E}}
}^2
\Bigg) 
%}
\end{align}
\janote{notas}
\begin{itemize}
\item para $t\gg 1$, $(1-r_z^2)/4 \approx 1/4$ en el régimen caótico 
($r_z\approx 0$)
\item para $t \gg 1$
el tercer término se hace approx 0 para el régimen caótico y distinto de 0 
para regular. De hecho, todos los términos, excepto, el 1/2, son comparables 
a tiempos largos para el integrable
\item a tiempos cortos el término que más contribuye es el término dentro la   
sumatoria del último término con índices $i=0$ y $j=1$
\end{itemize}

\newpage

\section{Purity of Choi matrix}
The quantum channel acting over the first $(i=1)$ spin is defined as:
\begin{equation}\label{eq:choi:chaometer:channel:2}
\mcE(\rho) = 
\Tr_E \qty(
U^\dagger(t)
\rho \otimes \dyad{\psi_0}
U(t)
),
\end{equation}
where $\ket{\psi_0} = \bigotimes_{i=2}^L \ket{\phi_i}$, with $\ket{\phi_i}$
a random state of a single particle, is the initial state of the environment.

To compute the Choi-\jami{} matrix $\mcD(t)$ of the quantum channel 
in~\eref{eq:choi:chaometer:channel:2} we use the definition 
$\qty(\mcE \otimes \one) \qty[\dyad{\phi^+}]$, with $\ket{\phi^+}=1/\sqrt{2}\qty(\ket{0,0} + \ket{1,1})$
the maximally entangled state between two spins, and obtain
\begin{align}\label{eq:choi:chaometer:2}
\mcD(t) &= 
\frac{1}{2}
\sum_{i,j,p,q}
\bra{j, \psi_{0E}} U^\dagger(t) \qty( \dyad{p}{q}\otimes \one_E ) U(t)
\ket{i, \psi_{0E}} \dyad{q,i}{p,j}.
\end{align}
After some algebraic step one may show that the purity of the Choi matrix is written
as:
\begin{equation}
\Tr[\mcD^2(t)] = 
\Tr( \Tr_S[U(t) (	\one_S \otimes \dyad{\psi_0}) U^\dagger(t)]^2 ).
\end{equation}
I believe that when the quantum channel acts over a $d$-dimensional system, 
and $U(t)=e^{-i H t}$, with $H$ a chaotic Hamiltonian (a matrix that can be taken 
from an appropriate RMT ensemble), the purity of the Choi matrix is equal to 
$1/d^2$ for $t\gg 1$ in the chaotic regime, and $1/d$ in the regular regime.

In particular, I would like to consider the spin chain Hamiltonian 
\begin{equation}\label{eq:H:wisniacki:ising:chain}
H = 
\sum_{i=1}^{L}
\qty(
h_x \sigma_i^x +
h_z \sigma_i^z
) -
\sum_{i=1}^{L-1}
J_z \sigma_{i}^z \sigma_{i+1}^z,
\end{equation}
for $h_x=j_i=1,\, \forall i$, $h_z=0.5$ (chaotic) and $h_z=2.5$ (regular).